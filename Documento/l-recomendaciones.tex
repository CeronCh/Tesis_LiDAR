\begin{itemize}
	\item Se sugiere la implementación de técnicas avanzadas de posprocesamiento, como el suavizado de datos o algoritmos de compensación geométrica, para reducir las dispersiones en las mediciones y mejorar la precisión en la reconstrucción de escenarios. Estas técnicas permitirían una mayor fidelidad en la representación de los entornos, optimizando la calidad de los modelos generados a partir de figuras primitivas y mejorando la eficiencia del sensor en entornos con variabilidad.
	\item Se propone explorar la incorporación de una rueda loca con bola en el diseño del montaje al agente robótico, ya que esta configuración reduce la fricción y ofrece una mayor libertad de movimiento. Esta mejora sería especialmente beneficiosa para el montaje del sensor en la parte frontal del agente, permitiendo una mayor estabilidad y flexibilidad en su desplazamiento, lo que podría optimizar el rendimiento del sensor y mejorar la maniobrabilidad del robot en diversos entornos.
	\item Se sugiere ampliar las funcionalidades de la herramienta de software para permitir la reconstrucción en tiempo real del entorno, eliminando la necesidad de intervención manual por parte del usuario.
	\item  Se recomienda continuar con la línea de investigación en el mapeo de entornos dinámicos, aprovechando los resultados obtenidos en este trabajo como base para futuras exploraciones. La implementación de nuevas técnicas de mapeo podrían expandir las capacidades del sensor y maximizar su rendimiento en entornos más complejos y variables.
\end{itemize}

