El alcance de este proyecto incluyó el desarrollo una aplicación capaz de capturar, procesar y visualizar las mediciones proporcionadas por el sensor LiDAR FHL-LD20, así como la realización de pruebas físicas para su caracterización y calibración. Estas tareas se llevaron a cabo utilizando herramientas de software como Python y MATLAB. Además, se generaron reconstrucciones precisas basadas en los datos obtenidos del sensor, las cuales permitieron evaluar su rendimiento y ajustar sus parámetros de medición.
	
De cara a proyectos futuros o trabajos relacionados con el mapeo de entornos y la navegación autónoma, la información recolectada por el sensor podrá integrarse a los algoritmos de navegación de robots móviles. Esto abrirá la posibilidad de implementar funcionalidades avanzadas, como la planificación de trayectorias para navegación punto a punto, optimizando así el desempeño de robots en aplicaciones como la exploración, la detección de obstáculos y el control autónomo en diversos entornos.