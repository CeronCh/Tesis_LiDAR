El alcance de este proyecto incluyó la selección, evaluación, calibración y caracterización de un sensor LiDAR para su futura integración en un sistema robótico autónomo, específicamente para aplicaciones de mapeo. Se desarrolló una herramienta de software que permitió la captura, procesamiento y visualización de los datos generados por el sensor, optimizando su uso en futuras tareas. Además, se definieron los criterios de evaluación y se realizaron pruebas en diversas condiciones para validar el desempeño del sensor en entornos controlados. El proyecto incluyó un análisis comparativo entre varios sensores, lo que facilitó la elección del más adecuado según los requisitos del sistema.

En cuanto a la calibración y caracterización del sensor, se diseñaron y ejecutaron pruebas controladas para determinar tanto el rango mínimo de medición como la varianza asociada a las mediciones de distancia y ángulo. Se utilizaron círculos de diferentes tamaños, con radios que variaban desde 60 mm hasta un máximo de 500 mm, para estudiar el comportamiento del sensor frente a distintas distancias y evaluar su efectividad. Las pruebas también incluyeron un análisis de la dispersión angular de las mediciones, lo que permitió evaluar el impacto de los cambios geométricos y la obstrucción visual sobre la precisión del sensor. Este análisis resultó clave para garantizar la fiabilidad de las mediciones en aplicaciones de navegación autónoma. Para ambos casos, se realizaron múltiples corridas variando los escenarios circulares con el fin de identificar patrones o tendencias específicas en el comportamiento del sistema.

Se desarrollaron herramientas de software en MATLAB y Python que facilitaron la captura, procesamiento y visualización de los datos del sensor. Estas herramientas fueron fundamentales para evaluar la repetibilidad y precisión de las mediciones en condiciones controladas, así como para la creación de la herramienta final destinada a la interpretación de los datos. No obstante, la comunicación del sensor se realizó exclusivamente mediante cables, lo que limitó la flexibilidad del sistema. Para futuras investigaciones, se sugiere explorar la incorporación de una transmisión de datos inalámbrica, como la que ofrece Wi-Fi, con el fin de mejorar la versatilidad del sistema y facilitar su integración en entornos más dinámicos.

Cabe destacar que esta investigación se centró en una distancia máxima de operación de 500 mm, acorde con los requerimientos de la aplicación específica. Aunque el sensor es capaz de medir distancias mayores, este aspecto no fue explorado durante el proyecto. A pesar de las limitaciones de tiempo que impidieron ampliar el rango de medición, los resultados obtenidos contribuyen significativamente a la base de conocimientos en el campo de la robótica, beneficiando tanto a la comunidad académica como profesional. Además, aunque se confirmó la viabilidad de integrar el sensor en agentes robóticos móviles, como el Pololu 3pi+, no se realizó una prueba física exhaustiva de los montajes propuestos. No obstante, esta investigación sienta las bases para que, en futuras fases del proyecto, dichos montajes se implementen físicamente, posibilitando el desarrollo de un vehículo explorador específicamente diseñado para el mapeo de entornos.





