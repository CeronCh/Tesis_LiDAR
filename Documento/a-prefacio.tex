Este trabajo es el resultado de un esfuerzo continuo, y no podría haberlo logrado sin el apoyo incondicional de las personas que siempre estuvieron a mi lado a lo largo de este proceso. A mi padre, Jorge, cuyo consejo y ejemplo me han guiado en cada paso, a mi madre, Brenda, por su amor y dedicación, que me ha enseñado a enfrentar los retos con paciencia y fortaleza, y a mi hermana, Camila, por ser mi apoyo emocional constante, saber cómo distraerme cuando lo necesitaba y, sobre todo, por su incansable cariño. Cada uno de ellos ha sido esencial para mi crecimiento personal y académico, y su presencia ha sido un pilar en mi vida durante toda esta carrera.

Quiero expresar también mi sincero agradecimiento a mi asesor de tesis, el Dr. Luis Alberto Rivera Estrada, quien ha sido una fuente invaluable de conocimiento y motivación. Su disposición constante para resolver mis dudas, su paciencia para guiarme en cada etapa de este proyecto y sus palabras de aliento me han permitido superar obstáculos y continuar avanzando. Sin su apoyo, este trabajo no habría sido posible.

A mis amigos y compañeros de estos cinco años, gracias por compartir este camino lleno de retos, aprendizajes y momentos inolvidables. Su amistad ha hecho que mi experiencia universitaria haya sido mucho más enriquecedora y memorable. También a mis amigos del colegio, quienes siempre han estado presentes, ofreciéndome su apoyo y comprensión a pesar de la distancia. Su apoyo constante es un tesoro que valoro profundamente.

Finalmente, mi agradecimiento se extiende a todos los profesores que, con generosidad, han compartido su conocimiento conmigo. Gracias por extender siempre su mano y respaldarme en cada paso del camino. Su apoyo no solo ha sido crucial para la realización de este trabajo de graduación, sino también para que pudiera enfrentar cada materia con confianza y éxito.