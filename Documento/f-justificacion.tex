En entornos desconocidos, los robots móviles deben ser capaces de desplazarse de un punto a otro de manera eficiente y segura tomando en cuenta los obstáculos y elementos de interés dentro del área de operación. En los trabajos de fases anteriores se han desarrollado e implementado, a nivel de simulación, algoritmos de exploración para el mapeo de entornos utilizando modelos virtuales de robots con tracción diferencial. El uso de estos algoritmos junto con las estimaciones del entorno han permitido que se evalúe el rendimiento de algoritmos para la planificación de trayectorias en espacios donde se desee llevar a cabo una navegación autónoma. Sin embargo, todos los avances que se han realizado han sido a nivel de simulación en software como Matlab y Webots, por lo que no se ha tenido una validación física de los algoritmos en robots móviles funcionales. 

Por esta razón, el enfoque principal de este trabajo es abordar, evaluar y seleccionar opciones de sensores de distancia que permitan la futura validación física de los algoritmos de exploración y mapeo de entornos en agentes robóticos móviles. Se busca verificar que, con los datos capturados del sensor, las estimaciones del espacio de trabajo sean consistentes con las mediciones del espacio en sí. Además, se propone buscar posibles formas de implementación para integrar los sensores de distancia a los robots diferenciales Pololu 3Pi+, lo cual dotaría a los agentes con capacidades adicionales. Esto establecería un precedente con buena base para futuras implementaciones en esta línea de investigación, como lo puede ser el mapeo de entornos físicos dinámicos.

El mapeo de entornos utilizando robots móviles resulta fundamental en una amplia gama de aplicaciones, desde la logística en el transporte de suministros en ambientes comerciales hasta la exploración de terrenos desconocidos en operaciones de búsqueda y rescate. La capacidad de generar mapas detallados que representen con precisión obstáculos y elementos relevantes, no solo mejora la comprensión del entorno sino que también establece una base sólida para la toma de decisiones y planificación de acciones en situaciones críticas, como desastres naturales. Estos mapas permiten identificar rutas seguras y evaluar riesgos en las operaciones, lo que mitiga los impactos negativos en situaciones de emergencia para los humanos.