Los avances en la tecnología de sensores y algoritmos de navegación han impulsado de manera significativa el desarrollo de robots móviles autónomos, los cuales son capaces de operar con alta eficiencia en entornos dinámicos y desconocidos. Estos sistemas tienen aplicaciones en diversas áreas, como la logística, la agricultura, la exploración espacial y las operaciones de rescate. Sin embargo, el éxito de estos robots depende en gran medida de su capacidad para percibir y modelar su entorno con precisión.

Este trabajo de investigación abordó la necesidad de evaluar y seleccionar sensores de distancia tipo láser adecuados para aplicaciones de mapeo en sistemas robóticos. Para ello, se analizaron tres modelos de sensores láser, considerando sus características técnicas y su posible integración con agentes robóticos dentro del ecosistema Robotat de la Universidad del Valle de Guatemala. Además, se desarrolló una herramienta de software que permite capturar, procesar y visualizar los datos obtenidos del sensor seleccionado, optimizando su uso para aplicaciones futuras. También se incluyeron pruebas de calibración y caracterización del sensor, en las cuales se establecieron criterios de comparación y evaluación para analizar su desempeño. Este enfoque permitió identificar posibles limitaciones y áreas de mejora, con el fin de optimizar el rendimiento del sensor y garantizar la fiabilidad de los datos obtenidos.

Un aspecto fundamental de esta investigación fue la selección y evaluación de los sensores de distancia. Para ello, se llevó a cabo un análisis comparativo entre tres modelos de sensores, con el objetivo de identificar cuál es el más adecuado para su integración futura en algoritmos de exploración y mapeo de entornos robóticos. Este análisis garantizó que la selección final maximizara el rendimiento del sensor, tomando en cuenta los criterios más relevantes para la aplicación.

En el desarrollo de la herramienta de software para la lectura e interpretación de los datos del sensor seleccionado, se detalló el formato de los paquetes de datos enviados por el sensor, así como su protocolo de comunicación y las características generales de su funcionamiento. Se hizo énfasis en el uso de software como MATLAB y Python para capturar y procesar esta información de manera eficiente. Además, se presentó la aplicación desarrollada, diseñada para facilitar la interacción del usuario con el sensor.

En la etapa de calibración del sensor, se detallaron las correcciones implementadas para compensar las desviaciones sistemáticas en las lecturas, garantizando que las mediciones reflejen con precisión las dimensiones reales del entorno. Este proceso permitió minimizar las discrepancias en la distancia y el ángulo, aspectos clave en las tareas de navegación robótica y mapeo. Además, se presentó un análisis de los criterios utilizados para evaluar la calibración del sensor. También se destacó el diseño de una plataforma física para las pruebas del sensor LiDAR, la cual incorpora paredes modulares que brindan flexibilidad para evaluar diferentes configuraciones del entorno.

Simultáneamente, se llevó a cabo la etapa de caracterización del sensor, evaluando su fiabilidad en condiciones reales de operación. Se analizaron las variaciones en las mediciones de distancia y ángulo, proporcionando datos sobre los parámetros de incertidumbre necesarios para integrar el sensor en algoritmos de navegación autónoma, como el filtro de Kalman extendido (EKF). Además, se detallaron las pruebas realizadas para evaluar la repetibilidad de los datos obtenidos y los análisis correspondientes. Las pruebas se realizaron en diversos entornos circulares controlados, con el fin de evaluar su rendimiento y la precisión de las mediciones en distintas longitudes.

Finalmente, se presentó una evaluación conceptual de la integración del sensor LiDAR FHL-LD20 en los robots de tracción diferencial Pololu 3pi+. Se incluyeron modelos 3D que ilustraron las posibles configuraciones del sensor, considerando el espacio disponible, la operatividad y las limitaciones inherentes a cada montaje. 