El objetivo principal de este trabajo fue evaluar tecnologías y opciones de sensores de distancia que puedan usarse para mapeo de entornos. Para ello, se llevó a cabo un análisis de las opciones disponibles en el mercado, considerando distintos criterios, y se seleccionó la mejor opción que se consideraba adecuada para su futura implementación con agentes robóticos móviles. Asimismo, se desarrolló una herramienta de software que permitió la comunicación con el sensor seleccionado, permitiendo adquirir, interpretar y visualizar los datos capturados de manera intuitiva y eficiente. 

Además, con esta herramienta, se caracterizó y calibró el LiDAR FHL-LD20, con el objetivo de optimizar su rendimiento y validar su precisión. Se llevaron a cabo pruebas en distintos entornos controlados para evaluar su funcionamiento y la fidelidad de sus mediciones bajo diferentes condiciones. Durante este proceso, se establecieron criterios de comparación y evaluación para analizar el desempeño del mismo. De esta manera, se identificaron posibles limitaciones y áreas de mejora, con el fin de optimizar su uso y garantizar fiabilidad de los datos obtenidos para su futura implementación en aplicaciones de mapeo y navegación en el ecosistema Robotat de la Universidad del Valle de Guatemala.