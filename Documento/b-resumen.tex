Este trabajo de investigación se enfocó en la evaluación de sensores de distancia tipo láser para aplicaciones de mapeo de entornos con agentes robóticos móviles, destacando la importancia de contar con sensores precisos y confiables para la navegación autónoma. A través de un análisis comparativo entre tres sensores de distancia tipo láser: VL53L0X, LiDAR FHL-LD20, y YDLIDAR Tmini Pro, se evaluaron diversos criterios y parámetros técnicos que permitieron seleccionar al sensor LiDAR FHL-LD20 como el más adecuado para su integración en dichas aplicaciones. Adicionalmente, se desarrolló una herramienta de software diseñada par procesar, visualizar e interpretar los datos generados por el sensor seleccionado. 

La herramienta no solo facilitó la interpretación de los datos, sino que también se consolidó como un recurso clave para validar el rendimiento del sensor en diversas condiciones operativas. De forma paralela, se llevaron a cabo pruebas de calibración y caracterización del sensor en escenarios específicos. Estas pruebas permitieron definir un rango mínimo de medición efectivo, depurar mediciones nulas, y analizar las distribuciones de las mediciones de distancia y ángulo, revelando que el sensor ofrece mediciones precisas, con desviaciones menores a 5 mm en las distancias y un margen de error inferior a 3° en las mediciones angulares.

Finalmente, se evaluó la viabilidad de integrar el sensor en agentes robóticos móviles disponibles en la Universidad.  Las contribuciones de esta investigación son significativas, ya que abren el camino para futuras aplicaciones en robótica autónoma y mapeo de entornos dinámicos, aportando herramientas útiles para avanzar en la integración de sensores LiDAR en sistemas robóticos móviles.